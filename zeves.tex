\begin{zed}
[DNI, NOMBRE, FECHA, HORA]
\end{zed}

\begin{schema}{Consultorio}
turnos: (FECHA \cross HORA) \pfun DNI \\
pacientes: DNI \pfun NOMBRE
\end{schema}

\begin{schema}{ConsultorioInit}
Consultorio
\where
turnos = \emptyset \\
pacientes = \emptyset
\end{schema}

\begin{schema}{ConsultorioInv}
Consultorio
\where
\ran turnos \subseteq \dom pacientes
\end{schema}

\begin{schema}{AltaPacienteOk}
\Delta Consultorio \\
d?: DNI \\
n?: NOMBRE
\where
d? \notin \dom pacientes \\
pacientes' = pacientes \cup \{d? \mapsto n?\} \\
turnos' = turnos
\end{schema}

\begin{schema}{PacienteYaExiste}
\Xi Consultorio \\
d?: DNI
\where
d? \in \dom pacientes
\end{schema}

\begin{zed}
AltaPaciente \defs AltaPacienteOk \lor PacienteYaExiste
\end{zed}

\begin{schema}{BajaPacienteOk}
\Delta Consultorio \\
d?: DNI
\where
d? \in \dom pacientes \\
d? \notin \ran turnos \\
pacientes' = \{d?\} \ndres pacientes \\
turnos' = turnos
\end{schema}

\begin{schema}{PacienteNoExiste}
\Xi Consultorio \\
d? : DNI
\where
d? \notin \dom pacientes
\end{schema}

\begin{schema}{PacienteTieneTurnos}
\Xi Consultorio \\
d? : DNI
\where
d? \in \ran turnos
\end{schema}

\begin{zed}
BajaPaciente \defs BajaPacienteOk \lor PacienteNoExiste \lor PacienteTieneTurnos
\end{zed}

\begin{schema}{AltaTurnoOk}
\Delta Consultorio \\
f? : FECHA \\
h? : HORA \\
d?: DNI
\where
d? \in \dom pacientes \\
(f?,h?) \notin \dom turnos \\
turnos' = turnos  \cup \{(f?,h?) \mapsto d?\} \\
pacientes' = pacientes
\end{schema}

\begin{schema}{TurnoYaAsignado}
\Xi Consultorio \\
f? : FECHA \\
h? : HORA
\where
(f?,h?) \in \dom turnos
\end{schema}

\begin{zed}
AltaTurno \defs AltaTurnoOk \lor PacienteNoExiste \lor TurnoYaAsignado
\end{zed}

\begin{schema}{BajaTurnoOk}
\Delta Consultorio \\
f? : FECHA \\
h? : HORA \\
d? : DNI
\where
(f?,h?) \in \dom turnos \\
d? = turnos\ (f?,h?) \\
turnos' = \{(f?,h?)\} \ndres turnos \\
pacientes' = pacientes
\end{schema}

\begin{schema}{TurnoNoAsignado}
\Xi Consultorio \\
f? : FECHA \\
h? : HORA
\where
(f?,h?) \notin \dom turnos
\end{schema}

\begin{schema}{TurnoNoCoincide}
\Xi Consultorio \\
f? : FECHA \\
h? : HORA \\
d? : DNI
\where
(f?,h?) \in \dom turnos \\
d? \neq turnos\ (f?,h?)
\end{schema}

\begin{zed}
BajaTurno \defs BajaTurnoOk \lor TurnoNoAsignado \lor TurnoNoCoincide
\end{zed}

\begin{schema}{TurnosPaciente}
\Xi Consultorio \\
d? : DNI \\
t! : FECHA \rel HORA
\where
t! = \dom (turnos \rres \{d?\})
\end{schema}

\begin{schema}{NombreTurnoOk}
\Xi Consultorio \\
f? : FECHA \\
h? : HORA \\
n! : NOMBRE
\where
(f?,h?) \in \dom turnos \\
n! = pacientes\ (turnos\ (f?,h?))
\end{schema}

\begin{zed}
NombreTurno \defs NombreTurnoOk \lor TurnoNoAsignado
\end{zed}

\begin{theorem}{AltaPacientePI}
ConsultorioInv \land AltaPaciente \implies ConsultorioInv'
\end{theorem}

\begin{zproof}[AltaPacientePI]
invoke AltaPaciente;
split AltaPacienteOk;
cases;
prove by reduce;
next;
prove by reduce;
next;
\end{zproof}

\begin{theorem}{AltaTurnoPI}
ConsultorioInv \land AltaTurno \implies ConsultorioInv'
\end{theorem}

\begin{zproof}[AltaTurnoPI]
invoke AltaTurno;
split AltaTurnoOk;
cases;
prove by reduce;
next;
prove by reduce;
apply inPower;
instantiate e == d?;
next;
\end{zproof}
